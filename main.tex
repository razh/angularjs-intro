\documentclass[12pt]{article}

\title{A simple introduction to \textsc{AngularJS}\date{}}
\author{Raymond Zhou}

\usepackage{booktabs}
\usepackage{helvet}

\usepackage{pgfplots}
\usepackage{tikz}
\pgfplotsset{compat=newest}
\usetikzlibrary{automata, shapes, shadows}

\usepackage{fancyhdr}
\pagestyle{fancy}

\setlength{\headheight}{15pt}

% Begin document.
\begin{document}

% Set up header and footer,
\fancyhf{}
\fancyhead[C]{$\hfill$A simple introduction to \textsc{AngularJS}}
\fancyfoot[C]{\thepage}
\renewcommand{\footrulewidth}{0.4pt}  % same width as the header line

\maketitle{}

\section{Introduction}

\textsc{AngularJS} approaches client-side web development in a radically different way from jQuery.
\\\\
This guide emphasizes readability over vocabulary and diagrams over paragraphs of technical detail to illustrate the mechanics of \textsc{AngularJS}.

\section{\texttt{\$scope}}

\begin{center}
  \begin{tabular}{cc}
    \toprule
      \textbf{scope} & \textbf{property}\tabularnewline
    \midrule
      root           & value \tabularnewline
      child          & test  \tabularnewline
      sibling        & id    \tabularnewline
    \bottomrule
  \end{tabular}
\end{center}

\begin{center}
  \begin{tikzpicture}

  \end{tikzpicture}
\end{center}

Possible topics of discussion.

\begin{itemize}
  \item scopes.
  \item Routing with ngViews.
  \item ngInclude.

\end{itemize}


\end{document}
