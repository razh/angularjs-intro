\documentclass[12pt]{article}

\title{A simple introduction to \textsc{AngularJS}\date{}}
\author{Raymond Zhou}

\usepackage{booktabs}
\usepackage{helvet}

\usepackage{pgfplots}
\usepackage{tikz}
\pgfplotsset{compat=newest}
\usetikzlibrary{automata, shapes, shadows}

\usepackage{fancyhdr}
\pagestyle{fancy}

\setlength{\headheight}{15pt}

% Begin document.
\begin{document}

% Set up header and footer,
\fancyhf{}
\fancyhead[C]{$\hfill$A simple introduction to \textsc{AngularJS}}
\fancyfoot[C]{\thepage}
\renewcommand{\footrulewidth}{0.4pt}  % same width as the header line

\maketitle{}

\section{Introduction}

\textsc{AngularJS} approaches client-side web development in a radically different way from jQuery.
\\\\
This guide emphasizes readability over vocabulary and diagrams over paragraphs of technical detail to illustrate the mechanics of \textsc{AngularJS}.

\section{\texttt{\$scope}}

\tikzset{
  every node/.style={inner sep=15pt}
}

\tikzstyle{split}=[draw, fill=white, rectangle split, rounded corners, drop shadow]
\begin{tikzpicture}
  \node (rootScope) [split] {
    \texttt{\large \$rootScope}
    \nodepart{two} {\texttt{\$scope.test}}
  };
\end{tikzpicture}

\tikzset {
  every two node part/.style={black},
}

\tikzstyle{split}=[draw, text=white, rectangle split, rectangle split part fill={black, white}, rounded corners]
\begin{tikzpicture}
  \node (rootScope) [split] {
    \texttt{\large \$rootScope}
    \nodepart{two} {\texttt{\$scope.test}}
    \nodepart{three} {\texttt{\$scope.test}}
    \nodepart{four} {\texttt{\$scope.test}}
  };
\end{tikzpicture}
\\\\
\begin{tabular}{cc}
  \toprule
    \textbf{scope} & \textbf{property}\tabularnewline
  \midrule
    root           & value \tabularnewline
    child          & test  \tabularnewline
    sibling        & id    \tabularnewline
  \bottomrule
\end{tabular}


\end{document}
